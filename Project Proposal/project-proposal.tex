\documentclass{ubicomp2012}
\usepackage{times}
\usepackage{url}
\usepackage{graphics}
\usepackage{color}
\usepackage[pdftex]{hyperref}
\hypersetup{%
pdftitle={iPlant}, pdfauthor={Jesper Sandberg and Thomas Kokholm}, pdfkeywords={iPlant, Inteligent Plant System, Pervasive, Arduino}, bookmarksnumbered, pdfstartview={FitH}, colorlinks,
citecolor=black, filecolor=black, linkcolor=black, urlcolor=black,
breaklinks=true, }
\newcommand{\comment}[1]{}
\definecolor{Orange}{rgb}{1,0.5,0}
\newcommand{\todo}[1]{\textsf{\textbf{\textcolor{Orange}{[[#1]]}}}}

%\pagenumbering{arabic}  % Arabic page numbers for submission.  Remove this line to eliminate page numbers for the camera ready copy

\begin{document}
% to make various LaTeX processors do the right thing with page size
\special{papersize=8.5in,11in}
\setlength{\paperheight}{11in}
\setlength{\paperwidth}{8.5in}
\setlength{\pdfpageheight}{\paperheight}
\setlength{\pdfpagewidth}{\paperwidth}

% use this command to override the default ACM copyright statement
% (e.g. for preprints). Remove for camera ready copy.
%\toappear{Submitted for review to UbiComp 2012.}



\title{iPlant: Automatic Plant System}
\subtitle{SPCL-2012 - Project Proposal}
\numberofauthors{2}
\author{
  \alignauthor Jesper Sandberg\\
    \affaddr{IT University of Copenhagen}\\
    \affaddr{Rued Langgaardsvej Vej 7}\\
    \affaddr{DK-2300 Copenhagen S}\\
    \email{jesan@itu.dk}
 \alignauthor Thomas Kokholm\\
    \affaddr{IT University of Copenhagen}\\
    \affaddr{Rued Langgaardsvej Vej 7}\\
    \affaddr{DK-2300 Copenhagen S}\\
    \email{tkok@itu.dk}  }
\maketitle

\section{Background and Motivation}
For years we have enjoyed the beauty and benefit of green plants. Yet people often struggle to keep their plants alive and fit. Plants require much attention and regular watering \& sunlight. Something that is easily forgotten in daily activities.



However plants are important for a healthy environment while they contribute to clean and natural air with the production of oxygen.
They help convert CO2 gasses and neutralize toxins in the air. (reference)

Our automatic plant system: iPlant will help attend plant(s) and provide users with information on temperature, humidity and general air quality near their plant(s). Using iPlant people can engage in having many different plants with absolute minimum effort. Even plants that require much attention like i.e. orchidaceae which are otherwise difficult to keep. \footnote{http://www.naturstyrelsen.dk/Naturbeskyttelse/Artsleksikon/Planter/Froeplanter/Blomsterplanter/orkideer.htm}

Furthermore users can combine multiple plants and monitor an entire villa or i.e. an office location with plants in different rooms connected to a local WiFi network. This ensure a natural work environment and provide all the information needed accessible from a simple web interface.

\section{Idea}
The main idea is to create a system that is capable of both watering and illuminate the plant(s). The system should be intelligent and capable of notifying the owner of the plant(s) with status information obtained through sensors placed directly within each plant.

Multiple plants can be connected to a local network using WiFi and monitored from a web interface.

Potting a plant combined with our solution provides the plant with the necessary attention for it to sustain on its own for longer periods of time, even when located with no access to sunlight. Our solution will automatically water the plant regularly - keeping a fixed level of humidity in the soil.

\section{Scenario}
A family leave their home for two weeks of much needed vacation. During their stay, their plants suffers from their absence.

In one room the curtains are close, and prevents the plants from getting sufficient sunlight - they stop growing and begin withering.

In another room plants are over-watered in the hope that they will survive during the absence. They unfortunately drown from the massive watering.

In a third room the family did not water their plants sufficiently and wither from regular exposure to sunlight without moist soil to drain from.

If only there was a system which could attend these plants, then the family would not only continue to have their plants upon return, they wouldn't ever have to worry when leaving home.

\section{Plan}
//main steps and milestones 

\section{Previous Works & Current Technologies}


\section{Requirements}
\begin{itemize}
\item Arduino microprocessor
\item Temperature Sensor
\item Humidity Sensor
\item Water Pump (Stepper motor)
\item Dust Sensor
\item Solar Panel / Sensor
\item UV-Diodes 
\end{itemize}

A prototype will be constructed and multiple tests will be performed and evaluate our solution. Exact specifications will follow along with a schedule project plan.

\section{Supervisor}
Sebastian B\"uttrich\\
IT University of Copenhagen\\
Rued Langgaardsvej Vej 7\\
DK-2300 Copenhagen S\\
sebastian@itu.dk

\nocite{example-journal,example-abstracts,example-conference2}

\bibliographystyle{abbrv}
\bibliography{sample}

\end{document}
