\documentclass{ubicomp2012}
\usepackage{times}
\usepackage{url}
\usepackage{graphics}
\usepackage{color}
\usepackage[pdftex]{hyperref}
\hypersetup{%
pdftitle={iPlant}, pdfauthor={Jesper Sandberg and Thomas Kokholm}, pdfkeywords={iPlant, Inteligent Plant System, Pervasive, Arduino}, bookmarksnumbered, pdfstartview={FitH}, colorlinks,
citecolor=black, filecolor=black, linkcolor=black, urlcolor=black,
breaklinks=true, }
\newcommand{\comment}[1]{}
\definecolor{Orange}{rgb}{1,0.5,0}
\newcommand{\todo}[1]{\textsf{\textbf{\textcolor{Orange}{[[#1]]}}}}

%\pagenumbering{arabic}  % Arabic page numbers for submission.  Remove this line to eliminate page numbers for the camera ready copy

\begin{document}
% to make various LaTeX processors do the right thing with page size
\special{papersize=8.5in,11in}
\setlength{\paperheight}{11in}
\setlength{\paperwidth}{8.5in}
\setlength{\pdfpageheight}{\paperheight}
\setlength{\pdfpagewidth}{\paperwidth}

% use this command to override the default ACM copyright statement
% (e.g. for preprints). Remove for camera ready copy.
%\toappear{Submitted for review to UbiComp 2012.}



\title{iPlant: Electronic Plant System}
\subtitle{SPCL-2012 - Project Proposal}
\numberofauthors{2}
\author{
  \alignauthor Jesper Sandberg\\
    \affaddr{IT University of Copenhagen}\\
    \affaddr{Rued Langgaardsvej Vej 7}\\
    \affaddr{DK-2300 Copenhagen S}\\
    \email{jesan@itu.dk}
 \alignauthor Thomas Kokholm\\
    \affaddr{IT University of Copenhagen}\\
    \affaddr{Rued Langgaardsvej Vej 7}\\
    \affaddr{DK-2300 Copenhagen S}\\
    \email{tkok@itu.dk}  }
\maketitle

\section{Background and Motivation}
For years we have enjoyed the beauty of green plants. Yet people often struggle to keep their plants fit. Plants require much attention and regular watering and sunlight is something that is easily overseen. (reference)

Even further some plants a more difficult to keep fit i.e. Orchidaceae, which easily dies when not treated daily.

 
 and other individuals have failed to keep plants alive, correct watering and sunlight is difficult to provide
in todays busy world. Yet we do desire to keep plants for their decoration as well as their oxygen production. A similar problem
is when is it necessary to vacuum, we do not wish to waste time by doing it too often, but still want our home to be clean,
today the typical solution is to make it a weekly tradition, but depending on how much you are at home and what you do, this is
far from an optimal solution. Our solution solves all these issues for the busy people of today, along with a huge range of
information about the temperature and other things in your home. Keep your home healthy.

\section{Idea}
The main idea is to create a system that is capable of both watering and illuminate plant(s). The system should be intelligent and capable of notifying the owner of the plant(s) with status information obtained trough sensors placed directly within the plant.

Potting a plant combined with our solution could make the plant sustain on its own, even when located without access regular sunlight. Our solution will be designed to automatically water the plant - keeping a fixed level of humidity in the soil.

\section{Scenario}
When traveling plants often suffer from the lack of attention. Or in some cases over-watering plants before going on holidays in the hope that they will survive for the time being.

Lorem ipsum dolor sit amet, consectetur adipiscing elit. Nulla sit amet libero neque. Integer dictum leo ac dui egestas scelerisque. Aliquam erat volutpat. Vivamus enim massa, rhoncus vel semper quis, egestas ut libero. Quisque cursus iaculis dictum. Quisque bibendum consequat aliquam. Sed sit amet lacus at lorem rutrum dapibus. Proin ac vulputate lectus. Duis sit amet lectus vel arcu ornare scelerisque ac ut quam. Cras rhoncus condimentum tempus. Donec sed mauris viverra magna convallis aliquet in et est. Ut porttitor sodales sapien eu dignissim. Nulla sit amet sapien et mauris luctus convallis. Pellentesque arcu urna, pharetra ac ullamcorper in, cursus et dolor.

\section{Plan}
Lorem ipsum dolor sit amet, consectetur adipiscing elit. Nulla sit amet libero neque. Integer dictum leo ac dui egestas scelerisque. Aliquam erat volutpat. Vivamus enim massa, rhoncus vel semper quis, egestas ut libero. Quisque cursus iaculis dictum. Quisque bibendum consequat aliquam. Sed sit amet lacus at lorem rutrum dapibus. Proin ac vulputate lectus. Duis sit amet lectus vel arcu ornare scelerisque ac ut quam. Cras rhoncus condimentum tempus. Donec sed mauris viverra magna convallis aliquet in et est. Ut porttitor sodales sapien eu dignissim. Nulla sit amet sapien et mauris luctus convallis. Pellentesque arcu urna, pharetra ac ullamcorper in, cursus et dolor.

\section{Requirements}
\begin{itemize}
\item Arduino microprocessor
\item Temperature Sensor
\item Humidity Sensor
\item Water Pump (Stepper motor)
\item Dust Sensor
\item Solar Panel / Sensor
\item UV-Diodes 
\end{itemize}

Lorem ipsum dolor sit amet, consectetur adipiscing elit. Nulla sit amet libero neque. Integer dictum leo ac dui egestas scelerisque. Aliquam erat volutpat. Vivamus enim massa, rhoncus vel semper quis, egestas ut libero. Quisque cursus iaculis dictum. Quisque bibendum consequat aliquam. Sed sit amet lacus at lorem rutrum dapibus.

\section{Supervisor}
Sebastian B\"uttrich\\
IT University of Copenhagen\\
Rued Langgaardsvej Vej 7\\
DK-2300 Copenhagen S\\
sebastian@itu.dk

\nocite{example-journal,example-abstracts,example-conference2}

\bibliographystyle{abbrv}
\bibliography{sample}

\end{document}
